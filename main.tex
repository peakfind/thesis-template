% Test document

\documentclass{mythesis}

\usepackage{lipsum}

\thesistype{中文学位论文模版}
\thesistitle{二维空间中麦克斯韦方程辐射条件及数值模拟}
\thesisauthor{张嘉译}
\thesisinstitute{数学科学学院}

\loadsymbols{symbols}

\begin{document}

\maketitle

\frontmatter

\begin{abstract}
    摘要测试
\end{abstract}

\tableofcontents

\mainmatter

\chapter{Introduction} \label{chp:intro}

This is a test document.

\lipsum[1-3]

\section{Background}

\lipsum[4-6]

\section{Problem formulation}

\lipsum[7-8]

\chapter{English version test}

We choose TeX as the default English font in our template.

\section{Test English typesetting}

We now focus on the half-line quasiperiodic problems. As these problems are very 
similar to each other, it is sufficient to study the half-line problem set on 
$(a, +\infty)$ and suppose without loss of generality that $a = 0$. Let 
$\mu_{\theta} := \mu_{p}(\theta \cdot)$ and $\rho_{\theta} := \rho_{p}(\theta \cdot)$. 
Therefore, the problem we consider in this section is the following:

Different font shapes
\begin{itemize}
    \item XCharter
    \item \textbf{XCharter}
    \item \textit{XCharter}
    \item \textsl{XCharter}
    \item \texttt{XCharter}
\end{itemize}

\section{Theorem environments}

\begin{theorem}[thm test]
   \lipsum[2] 
\end{theorem}

\begin{definition}[def test]
   \lipsum[2] 
\end{definition}

\begin{lemma}[lem test]
    \lipsum[2]
\end{lemma}

\begin{proposition}[prop test]
   \lipsum[2] 
\end{proposition}

\begin{Proof}
   \lipsum[2] 
\end{Proof}

\chapter{中文文档简介}

在本章中我们会介绍本模版的中文支持。

\section{中文字体}

目前本模版使用思源宋体作为默认字体。我们还定义了思源宋体的不同粗细
\begin{itemize}
    \item 常规: 思源宋体 
    \item 粗: \textbf{思源宋体}
    \item 斜体: \textit{思源宋体}
\end{itemize}

我们同样定义了{\heiti 思源黑体} 查看后面的内容是否受到黑体命令的影响。

我们使用了xcolor宏包。并在hyperref的设置中为不同的链接设定了不同的颜色。其效果如下:
在\autoref{chp:intro}中, 我们主要展示了本模版的英文排版效果。我的Github仓库为\url{https://github.com/peakfind}。

这是English中文混排Example。这是英文Chinese混排例子.

\chapter{Other functions}

\section{Symbol list}

The position \gls{x} and velocity \gls{v}. We also introduce the quasi momentum \gls{alpha}.

The position $x$ and velocity $v$. We also introduce the quasi momentum $\alpha$.

% Print the symbols
% \printlistofsymbols
\printnoidxglossaries
\end{document}